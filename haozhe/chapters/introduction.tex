\chapter{Introduction}

% assume factorised representations are introduced in the introduction chapter and the key of the factorised representations are sharing based on the conditional independencies. (similar to the sumprod)
% suggest: factorisation => branching => succinct representation


% Introduction Chapter

% Introduce join operations
% \begin{itemize}
%   \item What are join operations. Introduce the join operations using an example (triangle join)
%   \item 
%   \item Another entry in the list
% \end{itemize}

% % \section{Joins}
% what are join operations. studied extensively

% an example of join (also for the later use)

% Two challenges of join

% 1.
% - As we shown in the report, the first one is that a lot of join algorithms are sub optimal
% - Better bounds for the size of the query results are proposed recently
% - and they find that there exists join algorithms with better time complexity
% - New trends: worst-case optimal join algorithms

% normal join
% triangle join (cyclic query) social query
% might introduce large intermediate result
% but if we consider the three tables together, the intermediate result can be avoid. the blow up can be avoided. 

% 2. 
% - Redundancy in the computing and storing of join results
% - large intermediate results
% - listing representations introduce redundancy in the data
% - factorised representations were studied to reduce the redundancy of computing and storing join results


% % \section{Aggregates}
% what are aggregates

% generalisation of join algorithms to aggregates

% Framework to generalise the applications of aggregates
% The efficient algorithms with low computational complexity (inside-out)

% Applications of aggregates e.g., DB, logic, probabilistic graphical models, matrix chain computation

% % Functional Aggregate Queries (FAQs) 

% % \section{Optimisation}
% In-database learning of regression and classification models

% From join to \textbf{aggregate} and \textbf{optimisation} problems (other field of the Computer Science) 


% Then, the structure of the report


 